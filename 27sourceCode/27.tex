\documentclass{article}

\usepackage{listings}
\usepackage{xcolor}
 
\renewcommand{\sfdefault}{phv}
\renewcommand{\lstlistingname}{CODE}
\usepackage{courier} 
\usepackage{caption}
\DeclareCaptionFont{white}{\color{white}}
\DeclareCaptionFormat{listing}{\colorbox[cmyk]{0.43, 0.35,0.35,0.01}{\parbox{\textwidth}{#1#2#3}}}
\captionsetup[lstlisting]{format=listing,justification=raggedright,labelfont=white,textfont=white, singlelinecheck=false, margin=0pt, font={sf,bf,footnotesize}}
\begin{document}
	
 
	\lstset{
		language=[ANSI]{C},
		backgroundcolor=\color{pink},
		basicstyle=\footnotesize,
 %		breakatwhitespace=false,
		breaklines=true,
%		captionpos=b,
		commentstyle=\color{olive},
		directivestyle=\color{blue},
		extendedchars=false,
		% frame=single,%shadowbox
		framerule=0pt,
		keywordstyle=\color{blue}\bfseries,
		morekeywords={*,define,*,include...},
		numbersep=5pt,
		rulesepcolor=\color{red!20!green!20!blue!20},
	%	showspaces=false,
		%showstringspaces=false,
		%showtabs=false,
		stepnumber=2,
		stringstyle=\color{purple},
		tabsize=4,
		title=\lstname
		}
		\lstset{flexiblecolumns}
		
\begin{lstlisting}[language=C,caption = C code]
	//by baowenbo
	/* hello world*/
	
	#include"stdio.h"
	int main(int argc,char * argv){
		printf("output a value\n");
	}
\end{lstlisting}

	\lstset{
		extendedchars=false,
		commentstyle=\color{olive},
		basicstyle=\footnotesize\ttfamily, % Standardschrift
		numbers=left,               % Ort der Zeilennummern
		numberstyle=\tiny,          % Stil der Zeilennummern
		stepnumber=2,               % Abstand zwischen den Zeilennummern
		numbersep=5pt,              % Abstand der Nummern zum Text
		rulesepcolor=\color{red!20!green!20!blue!20},
		tabsize=4,                  % Groesse von Tabs
		extendedchars=true,         %
		breaklines=true,            % Zeilen werden Umgebrochen
		keywordstyle=\color{red},
		frame=b,         
		stringstyle=\color{purple}\ttfamily, % Farbe der String
		showspaces=false,           % Leerzeichen anzeigen ?
		showtabs=false,             % Tabs anzeigen ?
		xleftmargin=17pt,
		framexleftmargin=17pt,
		framexrightmargin=5pt,
		framexbottommargin=4pt,
		backgroundcolor=\color{pink},
		keywordstyle=\color{blue}\bfseries,
		morekeywords={*,define,*,include...}
		showstringspaces=false      % Leerzeichen in Strings anzeigen ?       
	}
\lstset{flexiblecolumns}
	 
\begin{lstlisting}[label=some-code,caption=Some Code,language=java]
	public void here() {
		goes().the().code()
	}
\end{lstlisting}
\lstinputlisting[label=samplecode,caption=A sample,language=java]{
		hello.java
}


\lstset{
	language=[ANSI]{C++},
	extendedchars=false,
	commentstyle=\color{olive},
	basicstyle=\footnotesize\ttfamily, % Standardschrift
	numbers=left,               % Ort der Zeilennummern
	numberstyle=\tiny,          % Stil der Zeilennummern
	stepnumber=2,               % Abstand zwischen den Zeilennummern
	numbersep=5pt,              % Abstand der Nummern zum Text
	rulesepcolor=\color{red!20!green!20!blue!20},
	tabsize=4,                  % Groesse von Tabs
	extendedchars=true,         %
	breaklines=true,            % Zeilen werden Umgebrochen
	keywordstyle=\color{red},
	frame=b,         
	stringstyle=\color{purple}\ttfamily, % Farbe der String
	showspaces=false,           % Leerzeichen anzeigen ?
	showtabs=false,             % Tabs anzeigen ?
	xleftmargin=17pt,
	framexleftmargin=17pt,
	framexrightmargin=5pt,
	framexbottommargin=4pt,
	backgroundcolor=\color{pink},
	keywordstyle=\color{blue}\bfseries,
	morekeywords={*,define,*,include...}
	showstringspaces=false      % Leerzeichen in Strings anzeigen ?       
}
\lstset{flexiblecolumns}
\lstset{morekeywords={dim3,__global__,__inline__,__device__}}
\lstinputlisting[label=my cuda code,caption=A sample]{
 my.cu
}

\end{document}